\documentclass[12pt]{article}
\usepackage[spanish]{babel}
\usepackage{amsmath}
\usepackage{graphicx}

\begin{document}

\begin{center}
\bf{\sc\Huge Ensayo Informatica 2}\\
\end{center}
\vspace{120pt}
\begin{center}
\bf{\sc\Huge Alumno}    
\end{center}
\begin{center}
\bf{\sc\Huge Juan manuel correa jiménez}\\
\end{center}
\vspace{200pt}
\begin{center}
\bf{\sc\Huge Universidad de antioquia}    
\end{center}
\begin{center}
\bf{\sc\Huge Medellín}
\end{center}
\begin{center}
\bf{\sc\Huge 2020}\\
\end{center}\
\newpage



\begin{center}

\bf{\sc\Huge El nacimiento de la computación }\\
\end{center}
\begin{flushleft}
\vspace{25PT}
\large
\section{Introducción}
""La comunicación por medio de computadores, celulares o cualquier otro método digital hace parte de la computación, al igual que los videojuegos, programas para calcular procesos matemáticos, etc. 	Ya que todos estos hacen un proceso finito y algorítmico. La computación tiene un campo de desarrollo demasiado amplio, y para poder llegar al límite de este faltan muchos años de desarrollo.

\vspace{15PT}
La computación ha existido desde hace muchos siglos atrás, pero la computación que utilizan las personas de hoy en día existe desde el siglo XX, todo gracias a una época llamada la crisis de los fundamentos la cual encaminó el nacimiento de la computación moderna.
\end{flushleft}
\newpage

\large
Al iniciar el siglo XX se creó la época llamada la crisis de los fundamentos, en la cual se hicieron grandes desarrollos y avances, como una nueva forma de hacer matemáticas. Y todo esto gracias a unos pocos que con mucho esfuerzo y sufrimiento trataron de introducir rigor lógico matemático.


\vspace{15PT}
El porqué de esta época se dio debido a un gran descubrimiento, y era que las matemáticas tenían contradicciones, por lo tanto se dio la conclusión de que las matemáticas no eran infalibles.

\vspace{15PT}
Este hecho desató muchas inquietudes por parte de los matemáticos de la época, y uno de estos matemáticos fue David Hilbert (1862-1943) el cual comenzó el “Programa Hilbert”, para demostrar que los sistemas axiomáticos podían ser infalibles con tres propiedades, la primera decía que podían ser consistentes (sin contradicciones), la segunda era, que siguiendo una serie de pasos algorítmicos se podía llegar a la solución y la última decía que tenían que ser completos. Y con esto darle solución al problema de las contradicciones provocadas en la crisis, pero en 1930 el lógico matemático Kurt Gödel (1906-1978) afirmó que el “Programa Hilbert” no se podía concluir. Al año siguiente Kurt demostró su primer teorema de incompletitud.

\vspace{15PT}
Pero aunque contradijo el “Programa Hilbert”, corroboró con que si el sistema está bien estructurado, este no demostrará si es verdadero o falso.

\vspace{15PT}
El matemático Alan Turing (1912-1954) siguió con el legado de Gödel, y este afirmó que no es posible determinar  si un problema escogido al azar tiene o no tiene solución.

\vspace{15PT}
 Además, demostró que hay problemas no computables, basándose en su máquina universal, la cual podía leer y escribir símbolos, y según él, podría resolver cualquier tarea algorítmica. En este punto de la historia Turing se convirtió en el padre de la computación moderna, ya que el estereotipo de su máquina universal era una máquina con capacidad de almacenamiento infinita. 
 
\vspace{15PT}
Esta idea fue la base de la computación moderna que actualmente se conoce, ya que impulsó a muchas personas a seguir con su legado y tratar de llegar a esa máquina infinita, actualmente todos los sistemas inteligentes que se utilizan hoy en día son máquinas que implementan varios grupos de máquinas de Turing.


\newpage
\section{Conclusión}
\large
Se puede decir que gracias a la crisis de los fundamentos, los matemáticos de la época pudieron abrir los ojos y los impulsó a buscar nuevas alternativas para dar una evolución al mundo de las matemáticas, tratando de demostrar que son infalibles frente a todo, y entre esta búsqueda lograron crear un mundo de infinitas posibilidades como lo es la computación como la conocemos hoy en día.  



\newpage
\section{Bibliografía}

*https://www.bbvaopenmind.com/ciencia/matematicas/asi-termino-el-sueno-de-las-matematicas-infalibles

\vspace{10PT}

*https://www.revistadelibros.com/articulos/javier-de-lorenzo-y-la-crisis-de-fundamentos-en-matematicas

\vspace{10PT}

*https://users.dcc.uchile.cl/~aabeliuk/documents/godel.pdf

\vspace{10PT}

*http://informatica.blogs.uoc.edu/2013/10/02/alan-turing-ii-el-nacimiento-de-la-computacion/

\vspace{10PT}

*http://www.cad.com.mx/historia-de-la-computacion.htm

 








\end{document}
